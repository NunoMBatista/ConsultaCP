\documentclass{article}
\usepackage[a4paper,landscape]{geometry}
\usepackage{multicol}
\usepackage{amsmath,amsthm,amsfonts,amssymb}
\usepackage{color,graphicx,overpic}
\usepackage{hyperref}
\usepackage{listings}

% Adjust the font size for the document
\renewcommand{\normalsize}{\footnotesize}

\lstset{
  basicstyle=\ttfamily\footnotesize, % Adjusted font size for code
  columns=fullflexible,
  frame=single,
  breaklines=true,
  postbreak=\mbox{\textcolor{red}{$\hookrightarrow$}\space},
}

\pdfinfo{
  /Title (Algorithms.pdf)
  /Creator (TeX)
  /Producer (pdfTeX 1.40.0)
  /Author (Tiago Silva)
  /Subject (C++ Algorithms)
  /Keywords (pdflatex, latex, pdftex, tex)}

% Set page margins for A4 paper in landscape mode
\geometry{top=1cm,left=1cm,right=1cm,bottom=1cm}

% Turn off header and footer
\pagestyle{empty}

% Redefine section commands to use less space
\makeatletter
\renewcommand{\section}{\@startsection{section}{1}{0mm}%
                                {-1ex plus -.5ex minus -.2ex}%
                                {0.5ex plus .2ex}%
                                {\normalfont\footnotesize\bfseries}}
\renewcommand{\subsection}{\@startsection{subsection}{2}{0mm}%
                                {-1ex plus -.5ex minus -.2ex}%
                                {0.5ex plus .2ex}%
                                {\normalfont\footnotesize\bfseries}}
\renewcommand{\subsubsection}{\@startsection{subsubsection}{3}{0mm}%
                                {-1ex plus -.5ex minus -.2ex}%
                                {1ex plus .2ex}%
                                {\normalfont\tiny\bfseries}}
\makeatother

% Define BibTeX command
\def\BibTeX{{\rm B\kern-.05em{\sc i\kern-.025em b}\kern-.08em
    T\kern-.1667em\lower.7ex\hbox{E}\kern-.125emX}}

% Don't print section numbers
\setcounter{secnumdepth}{0}

% Adjust paragraph formatting
\setlength{\parindent}{0pt}
\setlength{\parskip}{0pt plus 0.5ex}

% Multicol parameters
\setlength{\premulticols}{1pt}
\setlength{\postmulticols}{1pt}
\setlength{\multicolsep}{1pt}
\setlength{\columnsep}{10pt} % Adjust column separation if needed

\begin{document}
\raggedright
\footnotesize

% Center title
\begin{center}
     \large{\underline{CLIONS - Tiago Silva, Nuno Batista, João Coelho}} \\
\end{center}

% Start two columns
%\begin{multicols}{1}

\large
\huge\textbf{Macros and Typedefs:}
\large
\begin{lstlisting}
// Macros
#define forn(i,e) for(ll i = 0; i < e; i++)
#define forsn(i,s,e) for(ll i = s; i < e; i++)
#define rforn(i,s) for(ll i = s; i >= 0; i--)
#define ln  "\\n"
#define mp make_pair
#define pb push_back
#define fi first
#define se second
#define all(x) (x).begin(), (x).end()
#define sz(x) ((ll)(x).size())
#define INF 2e9

// Typedefs
typedef long long ll;
typedef long double ld;
typedef pair<int,int> pii;
typedef pair<ll,ll> pll;
typedef vector<ll> vll;
typedef vector<int> vi;
typedef vector<bool> vb;
typedef vector<vector<int>> vv;
typedef vector<pll> vpll;
\end{lstlisting}

\large
\huge\textbf{Main Function:}
\large
\begin{lstlisting}
int main() {
    ios_base::sync_with_stdio(false);
    cin.tie(NULL); 
    cout.tie(NULL);

    ll t;
    cin >> t;
    for (ll i = 0; i < t; i++) {
        solve();
    }
    return 0;
}
\end{lstlisting}

\large
\huge\textbf{Modular Arithmetic:}
\large
\begin{lstlisting}
// Modular function to avoid negative results
inline int mod(int a, int m) {
    return ((a % m) + m) % m;
}
\end{lstlisting}

\large
\huge\textbf{Longest Increasing Subsequence:}
\large
\begin{lstlisting}
// LIS algorithm with backtracking to print the sequence
void printLIS(int i, vi &p, vi &arr) {
    if (p[i] == -1) {
        cout << arr[i];
        return;
    }
    printLIS(p[i], p, arr);
    cout << ' ' << arr[i];
}

pii LIS(int n, vi &p, vi &arr) {
    int k = 0, lis_end = 0;
    vi L(n, 0), L_id(n, 0);
    p.assign(n, -1);

    for (int i = 0; i < n; i++) {
        int pos = lower_bound(L.begin(), L.begin() + k, arr[i]) - L.begin();
        L[pos] = arr[i];
        L_id[pos] = i;
        p[i] = pos ? L_id[pos-1] : -1;
        if (pos == k) {
            k = pos + 1;
            lis_end = i;
        }
    }
    return mp(k, lis_end);
}
\end{lstlisting}

\large
\huge\textbf{Coin Change (Bottom-Up):}
\large
\begin{lstlisting}
int CoinChangeBU(int nMoe, int troco, vi &moedas){
	vv trocos(nMoe+1, vi(troco+1, -1));
	for (int j = 0; j < troco+1; j++){
		trocos[0][j] = 0;
	}
	for (int j = 0; j <= nMoe; j++){
		trocos[j][0] = 1;
	}
	for (int j = 1; j <= nMoe; j++){
		for (int k = 1; k <= troco; k++){
			if (k < moedas[j-1]){
				trocos[j][k] = trocos[j-1][k];
			}else{
				trocos[j][k] = trocos[j-1][k] + trocos[j][k - moedas[j-1]];
			}
		}
	}
	return trocos[nMoe][troco];
}

int CoinChange1D(vi &moedas, int troco){
	vi trocos(troco + 1, 0);
	trocos[0] = 1;
	for (int j = 1; j <= troco; j++){
		for (int coin: moedas){
			if (j - coin >= 0){
				trocos[j] = (trocos[j] + trocos[j-coin])%MOD;
			}
		}
	}
	return trocos[troco]%MOD;
}

int LeastCoins(vi &moedas, int troco){
	vi trocos(troco + 1, INF);
	trocos[0] = 0;
	for (int j = 1; j <= troco; j++){
		for (int coin: moedas){
			if (j - coin >= 0){
				trocos[j] = min(trocos[j], trocos[j-coin] + 1);
			}
		}
	}
	if (trocos[troco] == INF){
		return -1;
	}else{
		return trocos[troco]%MOD;
	}
}
\end{lstlisting}

\large
\huge\textbf{Monotonic paths:}
\large
\begin{lstlisting}
//n e m arestas, NAO vertices
ll Monotonic(int n, int m, ll p){				//Se for n*n, usar mod(Catalan(n)*(n+1), m)
	n++;
	m++;
	vvll T(n, vll(m));
	for (int i = 0; i < n; i++){
	T[i][0] = 1;
	}
	for (int i = 0; i < m; i++){
	T[0][i] = 1;
	}
	for (int i = 1; i < n; i++){
		for (int j = 1; j < n; j++){
			T[i][j] = mod(mod(T[i-1][j], p) + mod(T[i][j-1], p), p);
		}
	}
	return mod(T[n-1][m-1], p);
}
\end{lstlisting}

\large
\huge\textbf{KS}
\large
\begin{lstlisting}
    int knapsack(int W, int n, vi price, vi weight){        //linear memory
	vi aux(W+1, 0);
	for (int i = 0; i < n; i++){
		for (int j = 1; j <= W; j++){
			if (weight[i] <= aux[j]){
				aux[j] = max(aux[j - weight[i]] + price[i], aux[j]);
			}
		}
	}
	return aux[W];
}
\end{lstlisting}

\large
\huge\textbf{Estruturas}
\large
\begin{lstlisting}
#include <ext/pb_ds/assoc_container.hpp>
#include <ext/pb_ds/tree_policy.hpp>
using namespace __gnu_pbds;

using oset = tree<int,         // key type
		null_type,   // value type
		less<int>,   // compare function
		rb_tree_tag,
		tree_order_statistics_node_update>;
auto s = oset();

#include <ext/pb_ds/assoc_container.hpp>
#include <ext/pb_ds/tree_policy.hpp>
using namespace __gnu_pbds;

using omap = tree<int,         // key type
		int,         // value type
		less<int>,   // compare function
		rb_tree_tag,
		tree_order_statistics_node_update>;
auto m = omap();

template<char MIN_CHAR = 'a', int ALPHABET = 26>
struct array_trie {
    struct trie_node {
        array<int, ALPHABET> child;
        int words_here = 0, starting_with = 0;

        trie_node() {
            memset(&child[0], -1, ALPHABET * sizeof(int));
        }
    };

    static const int ROOT = 0;

    vector<trie_node> nodes = {trie_node()};

    array_trie(int total_length = -1) {
        if (total_length >= 0)
            nodes.reserve(total_length + 1);
    }

    int get_or_create_child(int node, int c) {
        if (nodes[node].child[c] < 0) {
            nodes[node].child[c] = int(nodes.size());
            nodes.emplace_back();
        }
        return nodes[node].child[c];
    }

    int build(const string &word, int delta) {
        int node = ROOT;
        for (char ch : word) {
            nodes[node].starting_with += delta;
            node = get_or_create_child(node, ch - MIN_CHAR);
        }
        nodes[node].starting_with += delta;
        return node;
    }

    int add(const string &word) {
        int node = build(word, +1);
        nodes[node].words_here++;
        return node;
    }

    int erase(const string &word) {
        int node = build(word, -1);
        nodes[node].words_here--;
        return node;
    }

    int find(const string &str) const {
        int node = ROOT;
        for (char ch : str) {
            node = nodes[node].child[ch - MIN_CHAR];
            if (node < 0)
                break;
        }
        return node;
    }

    int count_prefixes(const string &str, bool include_full) const {
        int node = ROOT, count = 0;
        for (char ch : str) {
            count += nodes[node].words_here;
            node = nodes[node].child[ch - MIN_CHAR];
            if (node < 0)
                break;
        }
        if (include_full && node >= 0)
            count += nodes[node].words_here;
        return count;
    }

    int count_starting_with(const string &str, bool include_full) const {
        int node = find(str);
        if (node < 0)
            return 0;
        return nodes[node].starting_with - (include_full ? 0 : nodes[node].words_here);
    }
};

#define op(l, r) l + r		//operação da segment tree
const ll inf = 1e9;
struct Node {
    Node *l = 0, *r = 0;
    ll lo, hi, mset = inf, madd = 0, val = -inf;
    Node(ll lo,ll hi):lo(lo),hi(hi){} // Large interval of -inf
    Node(vi& v, ll lo, ll hi) : lo(lo), hi(hi) {
        if (lo + 1 < hi) {
            ll mid = lo + (hi - lo)/2;
            l = new Node(v, lo, mid); r = new Node(v, mid, hi);
            val = max(l->val, r->val);
        }
        else val = v[lo];
    }
    ll query(ll L, ll R) {
        if (R <= lo || hi <= L) return -inf;
        if (L <= lo && hi <= R) return val;
        push();
        return max(l->query(L, R), r->query(L, R));
    }
    void set(ll L, ll R, ll x) {
        if (R <= lo || hi <= L) return;
        if (L <= lo && hi <= R) mset = val = x, madd = 0;
        else {
            push(), l->set(L, R, x), r->set(L, R, x);
            val = max(l->val, r->val);
        }
    }
    void add(ll L, ll R, ll x) {
        if (R <= lo || hi <= L) return;
        if (L <= lo && hi <= R) {
            if (mset != inf) mset += x;
            else madd += x;
            val += x;
        }
        else {
            push(), l->add(L, R, x), r->add(L, R, x);
            val = max(l->val, r->val);
        }
    }
    void push() {
        if (!l) {
            ll mid = lo + (hi - lo)/2;
            l = new Node(lo, mid); r = new Node(mid, hi);
        }
        if (mset != inf)
            l->set(lo,hi,mset), r->set(lo,hi,mset), mset = inf;
        else if (madd)
            l->add(lo,hi,madd), r->add(lo,hi,madd), madd = 0;
    }
};

// Node for lowercase strings
struct Node {
	array<shared_ptr<Node>, 26> children;
	bool end;     // whether this node represents the end of a key
	size_t count; // optional (depending on queries)

	Node() : children{}, end{false}, count{0}  {}
};

class Trie {
private:
	shared_ptr<Node> root;
	explicit Trie(shared_ptr<Node> root) : root(root) {}

public:
	Trie() : root(new Node()) {}
	size_t size() const {
	return root->count;
	}

	bool exists(string_view s) const {
		auto node = root;
		for (auto c : s) {
			auto idx = c - 'a';
			if (node->children[idx]) {
				node = node->children[idx];
			} else {
				return false;
			}
		}
		return node->end;
	}

	optional<Trie> insert(string_view s) {
		if (exists(s)) {
			return {};
		}

		auto nroot = make_shared<Node>(*root);
		auto node = nroot;
		node->count += 1;
		for (auto c : s) {
			auto idx = c - 'a';
			if (node->children[idx]) {
				node = node->children[idx] = make_shared<Node>(*(node->children[idx]));
			} else {
				node = node->children[idx] = make_shared<Node>();
			}
			node->count += 1;
		}
		node->end = true;
		return Trie(nroot);
	}

	size_t count(string_view prefix) const {
		auto node = root.get();
		for (auto c : prefix) {
			auto idx = c - 'a';
			if (node->children[idx]) {
				node = node->children[idx].get();
			} else {
				return 0;
			}
		}
		return node->count;
	}
};

struct Node{ int mn, l, r; };

int init(int l, int r, Node st[], int* curr){
	if (l == r){ st[++(*curr)].mn = INF; return (*curr); }
	int m = l+(r-l)/2;
	int p = ++(*curr);
	st[p] = {0, init(l, m, st, curr), init(m+1, r, st, curr)};
	st[p].mn = min(st[st[p].l].mn, st[st[p].r].mn);
	return p;
}

int update(int i, int l, int r, int k, int x, Node st[], int* curr){
	if (l == r){ st[++(*curr)].mn = x; return *curr; }
	int m = l+(r-l)/2;
	int p = ++(*curr);
	if (k <= m){
		st[p] = {0, update(st[i].l, l, m, k, x, st, curr), st[i].r};
	} else {
		st[p] = {0, st[i].l, update(st[i].r, m+1, r, k, x, st, curr)};
	}
	st[p].mn = min(st[st[p].l].mn, st[st[p].r].mn);
	return p;
}

int query(int i, int l, int r, int tl, int tr, Node st[]){
	if (l > tr || r < tl) return INF;
	if (tl <= l && r <= tr) return st[i].mn;
	int m = l+(r-l)/2;
	return min(query(st[i].l, l, m, tl, tr, st), query(st[i].r, m+1, r, tl, tr, st));
}

int arr[n+1], root[n+2], curr = 1;				//Tres linhas seguintes por no solve
map<int, int> pos;
Node st[22*n];  
\end{lstlisting}

\large
\huge\textbf{Graphs}
\large
\begin{lstlisting}
vector<int> dijkstra(vector<vector<pii>>& graph, int source, int target) {
	int n = graph.size();
	vector<int> dist(n, INF);
	vector<bool> visited(n, false);
	dist[source] = 0;
	priority_queue<pii, vector<pii>, greater<pii>> pq;
	pq.push(make_pair(0, source));
	while (!pq.empty()) {
		int u = pq.top().second;
		pq.pop();
		if (visited[u]) {
			continue;
		}
		visited[u] = true;
		if (u == target) {
			break;
		}
		for (auto& neighbor : graph[u]) {
			int v = neighbor.first;
			int weight = neighbor.second;
			if (dist[v] > dist[u] + weight) {
				dist[v] = dist[u] + weight;
				pq.push(make_pair(dist[v], v));
			}
		}
	}
	return dist;
}

//Cada valor começa por ser o seu proprio set
void makeSet(int v, vi &parent) {
	parent[v] = v;
}
int findSet(int v, vi &parent) {
	if (v != parent[v])
	parent[v] = findSet(parent[v], parent);
	return parent[v];
}
void unionSets(int u, int v, vi &parent) {
	int root1 = findSet(u, parent);
	int root2 = findSet(v, parent);
	parent[root2] = root1;
}
bool check(int u, int v, vi &parent) {
	return findSet(u, parent) == findSet(v, parent);
}

void dfs (int v, vector<bool> &visited, vv &graph){
	visited[v] = true;
	for(int no: graph[v]){
		if (!visited[no]){
			dfs(no, visited, graph);
		}
	}
	return;
}

vector<vector<int>> adj(1001);
vector<bool> on_stack(1001);
vector<int> dfs(1001, 0);
vector<int> low(1001, -1);
int t = 1;
int c = 0;
stack<int> S;
void Tarjan(int v){
	low[v] = dfs[v] = t++;
	S.push(v);
	on_stack[v] = true;
	for(auto nbr:adj[v]){
		if(dfs[nbr] == 0){
			Tarjan(nbr);
			low[v] = min(low[v],low[nbr]);
		}
		else if(on_stack[nbr] == true){
			low[v] = min(low[v],dfs[nbr]);
		}
	}
	if (low[v] == dfs[v]){
		int nbr;
		do {
			nbr = S.top();
			S.pop();
			on_stack[nbr] = false;
		} while (nbr != v);
		c++;
	}
}

void BFS(int v, vv &graph){
	vector<bool> visited((int) graph.size(), false);
	visited[v] = true;
	queue<int> q;
	q.push(v);
	while (!q.empty()){
		int u = q.front();
		q.pop();
		if (!visited[u]){
			visited[u] = true;
			for(int s: graph[u]){
				q.push(s);
			}
		}
	}
}

#define rep(i, a, b) for(int i = a; i < (b); ++i)
template<class T> T edmondsKarp(vector<unordered_map<int, T>>&graph, int source, int sink) {
	assert(source != sink);
	T flow = 0;
	vi par(sz(graph)), q = par;

	for (;;) {
		fill(all(par), -1);
		par[source] = 0;
		int ptr = 1;
		q[0] = source;

		rep(i,0,ptr) {
			int x = q[i];
			for (auto e : graph[x]) {
				if (par[e.first] == -1 && e.second > 0) {
				par[e.first] = x;
				q[ptr++] = e.first;
				if (e.first == sink) goto out;
				}
			}
		}
		return flow;
out:
		T inc = numeric_limits<T>::max();
		for (int y = sink; y != source; y = par[y])
			inc = min(inc, graph[par[y]][y]);

		flow += inc;
		for (int y = sink; y != source; y = par[y]) {
			int p = par[y];
			if ((graph[p][y] -= inc) <= 0) graph[p].erase(y);
			graph[y][p] += inc;
		}
	}
}

typedef tuple<int, ll, ll, ll> edge;
class min_cost_max_flow {
private:
	int V;
	ll total_cost;
	vector<edge> EL;
	vector<vi> AL;
	vll d;
	vi last, vis;

	bool SPFA(int s, int t) { // SPFA to find augmenting path in residual graph
		d.assign(V, INF); d[s] = 0; vis[s] = 1;
		queue<int> q({s});
		while (!q.empty()) {
			int u = q.front(); q.pop(); vis[u] = 0;
			for (auto &idx : AL[u]) {                  // explore neighbors of u
				auto &[v, cap, flow, cost] = EL[idx];          // stored in EL[idx]
				if ((cap-flow > 0) && (d[v] > d[u] + cost)) {      // positive residual edge
					d[v] = d[u]+cost;
					if(!vis[v]) q.push(v), vis[v] = 1;
				}
			}
		}
		return d[t] != INF;                           // has an augmenting path
	}

	ll DFS(int u, int t, ll f = INF) {             // traverse from s->t
		if ((u == t) || (f == 0)) return f;
		vis[u] = 1;
		for (int &i = last[u]; i < (int)AL[u].size(); ++i) { // from last edge
			auto &[v, cap, flow, cost] = EL[AL[u][i]];
			if (!vis[v] && d[v] == d[u]+cost) {                      // in current layer graph
				if (ll pushed = DFS(v, t, min(f, cap-flow))) {
					total_cost += pushed * cost;
					flow += pushed;
					auto &[rv, rcap, rflow, rcost] = EL[AL[u][i]^1]; // back edge
					rflow -= pushed;
					vis[u] = 0;
					return pushed;
				}
			}
		}
		vis[u] = 0;
		return 0;
	}

public:
	min_cost_max_flow(int initialV) : V(initialV), total_cost(0) {
		EL.clear();
		AL.assign(V, vi());
		vis.assign(V, 0);
	}

	// if you are adding a bidirectional edge u<->v with weight w into your
	// flow graph, set directed = false (default value is directed = true)
	void add_edge(int u, int v, ll w, ll c, bool directed = true) {
		if (u == v) return;                          // safeguard: no self loop
		EL.emplace_back(v, w, 0, c);                 // u->v, cap w, flow 0, cost c
		AL[u].push_back(EL.size()-1);                // remember this index
		EL.emplace_back(u, 0, 0, -c);                // back edge
		AL[v].push_back(EL.size()-1);                // remember this index
		if (!directed) add_edge(v, u, w, c);         // add again in reverse
	}

	pair<ll, ll> mcmf(int s, int t) {
		ll mf = 0;                                   // mf stands for max_flow
		while (SPFA(s, t)) {                          // an O(V^2*E) algorithm
			last.assign(V, 0);                         // important speedup
			while (ll f = DFS(s, t))                   // exhaust blocking flow
				mf += f;
		}
		return {mf, total_cost};
	}
};

void solve(){
	int v, e, s, t;
	cin>>v>>e>>s>>t;
	min_cost_max_flow mf(v);
	for (int i = 0; i < e; i++){
		int a, b, cap, cost;
		cin>>a>>b>>cap>>cost;
		mf.add_edge(a, b, cap, cost);
	}
	pll res = mf.mcmf(s, t);
	cout<<res.fi<<' '<<res.se<<endl;
}

void AP(int v, vv &adj, vb &check, vi &dfs, vi &low, vi &parent, int &t, int &c){
	low[v] = dfs[v] = t++;
	for (auto nbr: adj[v]){
		if (dfs[nbr] == 0){
			parent[nbr] = v;
			AP(nbr, adj, check, dfs, low, parent, t, c);
			low[v] = min(low[v], low[nbr]);
			if (!check[v]){
				if (dfs[v] == 1){
					if (dfs[nbr] != 2) c++;
				}else{
					if (low[nbr] >= dfs[v]) c++;
				}
			}
			check[v] = true;
		}else if (parent[v] != nbr){
			low[v] = min(low[v], dfs[nbr]);
		}
	}
}

void solve(){
	int n, m;
	cin>>n>>m;
	vv adj(n+1);
	vb check(n+1);
	vi dfs(n+1, 0);
	vi low(n+1, -1);
	vi parent(n+1, -1);
	int t = 1;
	int c = 0;
	AP(1, adj, check, dfs, low, parent, t, c);
}

vi BF(vvpii &graph, int source){         //Codigo errado
	int n = graph.size();
	vi dist(n+1, INF);
	dist[source] = 0;
	for (int i = 1; i < n; i++){
		for (int j = 1; j < n; j++){
			for (auto nbr: graph[j]){
				int v = nbr.fi;
				int weight = nbr.se;
				if (dist[v] > dist[j] + weight){
					dist[v] = dist[j] +weight;
				}
			}
		}
	}
	for (int i = 1; i < n; i++){
		for (auto nbr: graph[i]){
			int v = nbr.fi;
			int weight = nbr.se;
			if (dist[v] > dist[i] + weight){
				flag = true;
			}
		}
	}
	return dist;
}

void FW(vi &matrix){
	int numVertices = (int) matrix.size();
	for (int k = 0; k < numVertices; k++){
		for (int i = 0; i < numVertices; i++){
			for (int j = 0; j < numVertices; j++){
				if (matrix[i][k] != INT_MAX && matrix[k][j] != INT_MAX){
					matrix[i][j] = min(matrix[i][j], matrix[i][k] + matrix[k][j]);
				}
			}
		}
	}
}

//Matematico

int f(int x){               //Avançar na expressão onde estamos a encontrar ciclo
	return (26*x + 11)%80;
}

pii floydCicleFinding(int x){             //Index (x) onde começa a sequencia (arr)
	int t = f(x), h = f(f(x));
	while (t != h){
		t = f(t);
		h = f(f(h));
	}
	int fase = 0, h = x;
	while (t != h){
		t = f(t);
		h = f(h);
		fase++;
	}
	int T = 1;
	h = f(t);
	while (t != h){
		h = f(h);
		T++;
	}
	return mp(T, fase);
}
\end{lstlisting}

\large
\huge\textbf{Modular/Matrices}
\large
\begin{lstlisting}
int modPow(int b, int p, int m){
	if (p == 0) return 1;
	int ans = modPow(b, p/2, m);
	ans = mod(ans*ans, m);
	if (p&1) ans = mod(ans*b, m);
	return ans;
}

vvll matMul(vvll &a, vvll &b, int MOD){              //Duas matrizes não nulas, i -> linhas, j -> colunas
	int lin = a.size();
	int col = b[0].size();
	vvll ans(lin, vll(col, 0));
	int par = b.size();
	for (int i = 0; i < lin; i++){
		for (int k = 0; k < par; k++){
			if (a[i][k] == 0) continue;
			for (int j = 0; j < col; j++){
				ans[i][j] += mod(a[i][k], MOD) * mod(b[k][j], MOD);
				ans[i][j] = mod(ans[i][j], MOD);
			}
		}
	}
	return ans;
}

vvll matPow(vvll base, int p, int MOD){          //So matrizes quadradas
	int lin = base.size();
	vvll ans(lin, vll(lin));
	for (int i = 0; i < lin; i++){
		for (int j = 0; j < lin; j++){
			ans[i][j] = (i == j);
		}
	}
	while (p){
		if (p&1){
			ans = matMul(ans, base, MOD);
		}
		base = matMul(base, base, MOD);
		p >>= 1;
	}
	return ans;
}

#define MAX_N 100  //adjust this value as needed
struct AugmentedMatrix{ double mat[MAX_N][MAX_N + 1];};
struct ColumnVector{ double vec[MAX_N];};
ColumnVector GaussianElimination(int N, AugmentedMatrix Aug){    //O(n³)
	//input: N, Augmented Matriz aug; output: Column Vector x, the answer
	for (int i = 0; i < N-1; i++){							//forward elimination
		int l = i;
		for (int j = i + 1; j < N; j++){										row with max col value
			if (fabs(Aug.mat[j][i]) > fabs(Aug.mat[l][i])) l = j;						remember this row l
		}
		//swap this pivot row, reason: minimize floating point error
		for (int k = i; k <= N; k++){
			swap(Aug.mat[i][k], Aug.mat[l][k]);
		}
		for (int j = i+1; j < N; j++){					//actual fwd elimination
			for (int k = N; k >= i; k--){
				Aug.mat[j][k] -= Aug.mat[i][k] * Aug.mat[j][i] / Aug.mat[i][i];
			}
		}
	}
	ColumnVector Ans;								//back substitution phase
	for (int j = N-1; j >= 0; j--){					//start from back
		double t = 0.0;
		for (int k = j+1; k < N; k++){
			t += Aug.mat[j][k] * Ans.vec[k];
		}
		Ans.vec[j] = (Aug.mat[j][N]-t) / Aug.mat[j][j];				//the answer is here
	}
	return Ans;
}

int main(){
	AugmentedMatrix Aug;
	Aug.mat[0][0] = 1; Aug.mat[0][1] = 1; Aug.mat[0][2] = 2; Aug.mat[0][3] = 9;	//x + y + 2z = 9
	Aug.mat[1][0] = 2; Aug.mat[1][1] = 4; Aug.mat[1][2] = 3; Aug.mat[1][3] = 1;	//2x + 4y - 3z = 1
	Aug.mat[2][0] = 3; Aug.mat[2][1] = 6; Aug.mat[2][2] = 5; Aug.mat[2][3] = 0;	//3x + 6y - 5z = 0
	ColumnVector X = GaussianElimination(3, Aug);
	cout<<"x = "<<X.vec[0]<<endl;
	cout<<"y = "<<X.vec[1]<<endl;
	cout<<"z = "<<X.vec[2]<<endl;
}
\end{lstlisting}

\large
\huge\textbf{Number Theory}
\large
\begin{lstlisting}
double raizN(double a, double N){             //(a)^(1/N)
	return pow(a, 1.0/N);
}

int countDigitos(double num, double baseNum, double baseNova){
	return floor(1 + log(num)/log(baseNova));
}

ll maxRangeSum1D(int n, vll &arr){
	ll ans = 0;
	//limpeza dos negativos
	ans = arr[0];
	for (int j = 0; j < n; j++){
		if (arr[j] >= 0){
			ans = 0;
			break;
		}else{
			if (arr[j] > ans) ans = arr[j];
		}
	}
	if (ans < 0) return ans;
	//fim de limpeza
	ans = 0;
	ll sum = 0;
	for (int j = 0; j < n; j++){
		sum += arr[j];
		ans = max(ans, sum);
		if (sum < 0) sum = 0;
	}
	return ans;
}

ll maxRangeSum2D(int n, vvll &mat){
	for (int i = 0; i < n; i++){
		for (int j = 1; j < n; j++){
			mat[i][j] += mat[i][j-1];
		}
	}
	ll maior = -INF;
	for (int i = 0; i < n; i++){
		for (int j = i; j < n; j++){
			ll subrect = 0;
			for (int k = 0; k < n; k++){
				if (i > 0) subrect += mat[k][j] - mat[k][i-1];
				else subrect += mat[k][j];
				if (subrect < 0) subrect = 0;
				maior = max(maior, subrect);
			}
		}
	}
	return maior;
}

//Primos

ll sieve_size;
bitset<10000010> bs;
vll p;

void gerador(ll upperbound){                //Não maior de 10^7
	sieve_size = upperbound+1;
	bs.set();
	bs[0] = bs[1] = 0;
	for (ll i = 0; i < sieve_size; i++){
		if (bs[i]){
			for (ll j = i*i; j < sieve_size; j+=i) bs[j] = 0;
			p.push_back(i);
		}
	}
}

bool isPrime(ll N){
	if (N < sieve_size) return bs[N];
	for (int i = 0; i < (int) p.size() && p[i]*p[i] <= N; i++){
		if (N%p[i] == 0) return false;
	}
	return true;
}

//Por no solve
gerador(10000000);

vll primeFactor(ll N){      //Fatorizar em numeros primos, não esquecer de gerar numeros primos
	vll factors;
	for (int i = 0; (i < (int) p.size()) && (p[i]*p[i] <= N); i++){
		while (N%p[i] == 0){
			N /= p[i];
			factors.pb(p[i]);
		}
	}
	if (N != 1) factors.pb(N);
	return factors;
}

int numFatPrimos(ll N){     //Quantos fatores primos tem um numero
	int ans = 1;
	for (int i = 0; (i < (int) p.size()) && (p[i]*p[i] <= N); i++){
		while (N%p[i] == 0) {
			N/=p[i];
			ans++;
		}
	}
	return ans + (N != 1);
}

int numDivisores(ll N){          //Multiplicatorio de (n+1), sendo 'n' o numero de vezes que cada fator primos aparece
	int ans = 0;
	for (int i = 0; (i < (int) p.size()) && (p[i]*p[i] <= N); i++){
		int power = 0;
		while (N%p[i] == 0){
			N /= p[i];
			++power;
		}
		ans += power+1;
	}
	return (N != 1) ? 2*ans : ans;
}

ll sumDivisores(ll N){          //Multiplicatório de (a^(n+1) - 1)/(a-1), sendo 'a' cada fator primo e 'n' o número de vezes que 'a' se repete
	ll ans = 1;
	for (int i = 0; (i < (int) p.size()) && (p[i]*p[i] <= N); i++){
		ll multiplier = p[i], total = 1;
		while (N%p[i] == 0){
			N /= p[i];
			total += multiplier;
			multiplier *= p[i];
		}
		ans *= total;
	}
	if (N != 1) ans *= (N+1);
	return ans;
}

ll numCoprimos(ll N){       //N * Multiplicatorio de (1 - 1/a), sendo 'a' cada fator primo de N
	ll ans = N;
	for (int i = 0; (i < (int) p.size()) && (p[i]*p[i] <= N); i++){
		if (N%p[i] == 0) ans -= ans/p[i];
		while (N%p[i] == 0) N/=p[i];
	}
	if (N != 1) ans -= ans/N;
	return ans;
}
 
vi numDiffFatPrimos(ll MAX_N){    //MAX_N <= 10^7       Numero de fatores primos diferentes para mt queries
	vi arr(MAX_N + 10, 0);
	for (int i = 2; i <= MAX_N; i++){
		if (arr[i] == 0){
			for (int j = i; j <= MAX_N; j+=i){
				++arr[j];
			}
		}
	}
	return arr;
}

vi numCoprimosMtQueries(ll Max_n){      //Max_n <= 10^7     Numero de coprimos para mt queries
	vi arr(Max_n);
	for (int i = 1; i <= Max_n; i++) arr[i] = i;
	for (int i = 2; i <= Max_n; i++){
		if (arr[i] == 0){
			for (int j = i; j <= Max_n; j+=i){
				arr[j] = (arr[j]/i) * (i-1);
			}
		}
	}
	return arr;
}

int extEuclidean(int a, int b, int &x, int &y){
	int xx = y = 0;
	int yy = x = 1;
	while (b){
		int q = a/b;
		int t = b;
		b = a%b;
		a = t;
		t = xx;
		xx = x-q*xx;
		x = t;
		t = yy;
		yy = y - q*yy;
		y = t;
	}
	return a;
}

int modInverse(int A, int M){          //Para combinações/fatoriais, escrever comb ou fatoriais
	int x, y
	int d = extEuclidean(A, M, x, y);
	if (d != 1) return -1;
	return mod(x, M);
}

pii diophantine(int a, int b, int sol){         //a*x + b*y = sol
	int x, y;
	int d = extEuclidean(a, b, x, y);           //gcd(a, b)    
	int mult = sol/d;
	x *= mult;
	y *= mult;
	b /= d;
	a /= d;
	int liminf = 0, limsup = INF;
	if ((x < 0) != (b < 0)){
		liminf = abs(x/b);
		if (x%b) liminf++;
	}else{
		limsup = abs(x/b);
	}
	if ((y < 0) != (a < 0)){
		int aux = abs(y/a);
		if (y%a) aux++;
		liminf = max(liminf, aux);
	}else{
		limsup = min(limsup, abs(y/a));
	}
	if (liminf > limsup) return mp(-1, -1);         //Só devolve uma solução para a equação, mas há um limite (finito ou infinito de soluções)
	else return mp(x + b*liminf, y + a*liminf);
}

int crt(vi &r, vi &m){
	int mt = accumulate(m.begin(), m.end(), 1, multiplies<>());
	int x = 0;
	for (int i = 0; i < (int) m.size(); i++){
		int a = mod((ll)r[i] * modInverse(mt/m[i], m[i]), m[i]);
		x = mod(x + (ll)a * (mt/m[i]), mt);
	}
	return x;
}

vll Catalan(int n, ll m){                  //n inclusive
	vll cat(n+1);
	cat[0] = 1;
	for (int i = 0; i < n; i++){
		cat[i+1] = mod(mod(mod((4*i)+2,m) * mod(cat[i],m)) * modInverse(i+2, m),);
	}
	return cat;
}

inline long long int gcd(int a, int b){
	while (b) {
		a %= b;
		swap(a, b);
	}
	return a;
}

inline long long int lcm (int a, int b){
	return (a / gcd(a, b)) * b;
}

int modInverse(int A, int M){
	int m0 = M;
	int y = 0, x = 1;

	if (M == 1)
		return 0;

	while (A > 1) {
		// q is quotient
		int q = A / M;
		int t = M;

		// m is remainder now, process same as
		// Euclid's algo
		M = A % M, A = t;
		t = y;

		// Update y and x
		y = x - q * y;
		x = t;
	}

	// Make x positive
	if (x < 0)
		x += m0;

	return x;
}

vpll fat;

void fatoriais(int tam, int m, vpll &res){
	res.pb(mp(1,1));
	for (int j = 1; j <= tam; j++){
		res.pb(mp((res[j-1].fi*j)%m, 0));
	}
	ll inv = modInverse(res[tam].fi, m);
	res[tam].se = inv;
	for (int j = tam-1; j > 0; j--){
		res[j].se = (res[j+1].se*(j+1))%m;
	}
}

ll comb(int c, int d, int m){
	if (d == 0) return 1;
	if ((d > 0) && (d > c)) return 0;
	return (((fat[c].fi*fat[d].se)%m)*fat[c-d].se)%m;
}
fatoriais(5000, MOD, fat);                                 //Colocar dentro da main

pll junta(pll a, pll b){
	if (a.fi < b.se && a.se > b.fi){
		return mp(max(a.fi, b.fi), min(a.se, b.se));
	}else{
		return mp(-1, -1);
	}
}

\end{lstlisting}

\large
\huge\textbf{Strings}
\large
\begin{lstlisting}
string text;    //Text
int n;          //Size of text,
int k;          //Number of keys
int maxs = 0;   // Should be equal to the sum of the length of all keywords.
int maxc = 26; // Maximum number of characters in input alphabet

// Returns the number of states that the built machine has.
// States are numbered 0 up to the return value - 1, inclusive .
int buildMatchingMachine(string arr[], int k, vector<map<int, bool>> &out, vi &f, vv &g){
	int states = 1;
	for ( int i = 0; i < k; ++i){ // Construct values for goto function, i .e ., fill g
		const string &word = arr[i];
		int currentState = 0;
		for ( int j = 0; j < (int) word.size(); ++j){
			int ch = word[j]-'a';
			if (g[currentState][ch] == -1){ // Allocate a new node (create a new state) if a node for ch doesnt exist .
				g[currentState][ch] = states++;
			}
			currentState = g[currentState][ch];
		}
		out[currentState][i] = true; // Add current word in output function
	}
	for ( int ch = 0; ch < maxc; ++ch){
		if (g[0][ch] == -1){
			g[0][ch] = 0;
		}
	}
	queue<int> q;
	for ( int ch = 0; ch < maxc; ++ch){
		if (g[0][ch] != 0){
			f [g[0][ch]] = 0;
			q.push(g[0][ch]) ;
		}
	}
	while (q.size () ) {
		int state = q.front () ;
		q.pop();
		for ( int ch = 0; ch < maxc; ++ch){
			if (g[state][ch] != -1){
				int failure = f [state];
				while (g[ failure][ch] == -1){ // Find the deepest node labeled by proper suffix of string from root to current state .
					failure = f [ failure ];
				}
				failure = g[failure][ch];
				f [g[state][ch]] = failure ;
				for (pair<int, bool> par: out[failure]){
					out[g[state][ch]][par.fi] = par.se;
				}
				q.push(g[state][ch]) ;
			}
		}
	}
	return states ;
}

int findNextState(int currentState, char nextInput, vector<map<int, bool>> &out, vi &f, vv &g){ //Returns the next state the machine will transition to using goto and failure functions.
	int answer = currentState;
	int ch = nextInput -'a';
	while (g[answer][ch] == -1){
		answer = f[answer];
	}
	return g[answer][ch];
}

void searchWords(string arr[], int k, string text, vector<map<int, bool>> &out, vi &f, vv &g, vv &ocor, vi &tam) {
	buildMatchingMachine(arr, k, out, f, g); // Build machine with goto, failure and output functions
	int currentState = 0;
	for ( int i = 0; i < (int) text.size() ; ++i){
		currentState = findNextState(currentState, text[i], out, f, g) ;
		/*if (out[currentState] == 0){ // If match not found, move to next state, uncomment if number of keys is less of 64
			continue;
		}*/
		for (pair<int, bool> par: out[currentState]){ // Match found, print all matching words of arr[]
			ocor[i-tam[par.fi]+1].pb(par.fi);
		}
	}
}

void solve(){
	cin>>text;
	n = (int) text.size();
	vv ocor(n);                     //To store the index where each key starts in texts
	cin>>k;
	string arr[k];                  //Stores every key
	vi tam(k);                      //Stores every key size
	for (int j = 0; j < k; j++){
		cin>>arr[j];
		tam[j] = arr[j].size();
		maxs += tam[j];
	}
	vector<map<int, bool>> out(maxs);                       // Stores the word number for each state (letter in text)
	//vi out(maxs, 0);                                      // Bit i in this mask is one if the word with index i in that state. To use if there are less than 64 keys
	vi f (maxs, -1);                                        // FAILURE FUNCTION IS IMPLEMENTED USING f[]
	vv g (maxs, vi(maxc, -1));                              // GOTO FUNCTION (OR TRIE) IS IMPLEMENTED USING g[][]
	searchWords(arr, k, text, out, f, g, ocor, tam);        // Each state (char in text) has the key numbers of the keys that start in that state in ocor  
	return;
}

string T, P;                    // T = text, P = pattern
int n, m;                       // n = |T|, m = |P|

void kmpPreprocess(vi &b) {                         // call this first
	int i = 0, j = -1; b[0] = -1;                   // starting values
	while (i < m) {                                 // pre-process P
	while ((j >= 0) && (P[i] != P[j])) j = b[j];    // different, reset j
		++i; ++j;                                   // same, advance both
		b[i] = j;
	}
}

void kmpSearch(vi &b) {                             // similar as above
	int i = 0, j = 0;                               // starting values
	while (i < n) {                                 // search through T
	while ((j >= 0) && (T[i] != P[j])) j = b[j];    // if different, reset j
		++i; ++j;                                   // if same, advance both
		if (j == m) {                               // a match is found
			printf("P is found at index %d in T\n", i-j);
			j = b[j];                               // prepare j for the next
		}
	}
}

void solve(){
	cin>>T;
	cin>>P;
	n = (int) T.size();
	m = (int) P.size();
	vi b(m+1);                // b = back table
	kmpPreprocess(b);
	kmpSearch(b);
}

int LCS(string a, string b, int tamA, int tamB){
	vv bu(tamA + 2, vi(tamB, 0));
	for (int i = 1; i <= tamA; i++){
		for (int j = 1; j <= tamB; j++){
			if (a[i-1] == b[j-1]) bu[i][j] = bu[i-1][j-1] + 1;
			else bu[i][j] = max(bu[i-1][j], bu[i][j-1]);
		}
	}
	return bu[tamA][tamB];
}
\end{lstlisting}

%\end{multicols}
\end{document}
